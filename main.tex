\documentclass[12pt, a4 paper]{article}
\usepackage{fullpage}
\usepackage{graphicx}
\usepackage{caption}
\usepackage{refstyle}
\usepackage{amsfonts}
\usepackage[left=1in,right=1in,top=1in,bottom=1in]{geometry}
\usepackage{balance} 
\usepackage[numbers]{natbib}
\usepackage{amsmath}
\usepackage{amssymb}
\usepackage{siunitx}
\usepackage{smartdiagram}
\usepackage{tikz}
\usepackage{mathabx}
\usepackage{MnSymbol}
\usetikzlibrary{shapes.geometric, arrows}
\usepackage{graphics}
\graphicspath{{/home/vasundhara/Desktop/}}
\usepackage{physics}
\bibliography{references}
\usepackage{float}
\usepackage{imakeidx}
\usepackage{hyperref}
\usepackage{siunitx}
%\usepackage{bibtex}
\usepackage[utf8]{inputenc}
\usepackage[english]{babel}
\documentclass{article}
\usepackage{amsmath}

\usepackage[dvipsnames]{xcolor}
\makeatletter
\DeclareTextCommand{\textprime}{\encodingdefault}{%
  \mbox{$\m@th'\kern-\scriptspace$}% 
  }
\makeatother
\begin{document}


\begin{titlepage}


\end{titlepage}
\tableofcontents
\newpage
\section{Introduction}
\subsubsection{Fermi Bubbles}

One of the big findings of the Fermi-LAT telescope were the Fermi bubbles which we had been oblivious to in the past. The Fermi bubbles look like two massive outbursts which appear to be originating from the galactic center,however, the origin of the bubbles are yet unknown. These bubbles span across roughly $17kpc$ and have a radii of about $3kpc$ \textcolor{red}{put reference here} .They appear to be symmetric when seen in gamma ray spectrum.\\

To understand our motivation to model the magnetic field of the fermi bubbles we first need to understand why is it essential to do so.
The galactic magnetic field have been always difficult to understand. Several efforts have been made to model galactic magnetic fields \textcolor{red}{put reference here}.In general the galactic magnetic field can be divided into two components,first a random turbulent component and second a structured component.
Most of the magnetic field models that currently exist incorporate both of these components and make further changes to them.\\

However, for our work we will focus only on the structured fields and completely ignore the effects of turbulent fields.
Up until now there have been a lot of magnetic field models proposed  \textcolor{red}{put reference here}.The magnetic field model that we took as a framework to build upon was the one given by Jansson and Farrar 2012. The JF12 model divides the galactic magnetic field in to $3$ structured components namely, disc field, toroidal halo field and the X-field. We studied each component of the JF12 field separately and this helped us better understand how the different component of the galactic magnetic field act. Since, in our case we were interested in only probing the halo part of the galaxy we have cut out the disc from our models completely. 

This paper has been divided in to $4$ sections $1)$ deals with our description of the Fermi bubble model, $2)$ talks about synchrotron polarisation,in section $3)$ we talk about our results and $4)$ will be a summary of the work.
\newpage
\section{Fermi Bubble Model}
The fermi bubble up until now have never been incorporated in the modelling of galactic magnetic field up until now. Polarisation data measured by SPASS \textcolor{red}{put Caretti reference here} show that the magnetic field in the fermi bubble could be around $6muG$ or $15muG$ depending on the kind of electron distribution \textcolor{red}{fill in details properly by Caretti}.This observation in itself was a motivation for us to explore the possiblity of adding fermi bubble in the existing JF12 model. 

There have been different descriptions for halo component of the galactic magnetic field. The JF12 model sets halo height at $5.3kpc$,for our work we set the height of the galactic halo at $5.3kpc$ similar to JF12 however, on top of that we had another halo component which is the Fermi bubble halo (FBHalo from now) and we set it at a height of $7kpc$ slightly lower than its actual height of $8kpc$ as seen by observation. The FBHalo field is defined as 



\[
     B_{FBtor}= exp(-\abs{z}/z_{0})\times 
\begin{cases}
    B_{0}* \exp(-z_{min}/\abs{z})*\exp(-\abs{r}/r_{0}),& \text{if } z\ge 0\\
    B_{0}* \exp(-z_{min}/\abs{z})*\exp(-\abs{r}/r_{0}),& \text{or else} 
\end{cases}
\]
The halo field has an exponential scale height $z{0}$ and $B_{0}$ is the amplitude of the north and south field. The norther and southern radial extent of the field is the same and is defined by $r_{0}$.
\begin{table}[h!]
  \begin{center}
    \caption{Parameter list for the Fermi bubble model}
    \label{tab:table1}
    
    \begin{tabular}{ |p{3cm}||p{3cm}|p{3cm}|} % <-- Alignments: 1st column left, 2nd middle and 3rd right, with vertical lines in between
    \hline
      \textbf{Field} & \textbf{Parameters} & \textbf{Description}\\
      \hline
      FBHalo & $B_{0}$ = $6muG$ & Field strength as measured in Caretti2013\\
       & $z_{0}$ = $7kpc$ & Height of the halo field for the fermi bubble beyond which field starts to drop\\
       & $r_{0}$ = $2kpc$ & Radial extent beyond which the field starts to drop c\\
       \hline
    \end{tabular}
  \end{center}
\end{table}


\subsubsection{Methodology}

\section{Synchrotron Polarisation - change words}
We genertaed synthetic synchrotron 
The kind of polarisation that we can observe are listed below.
\begin{enumerate}

    \item We know that free-free emission or thermal bremsstrahlung, is emitted by free electrons interacting with ions,in ionised gas. is intrinsically unpolarised ([Rybicki \& Lightman] 1979). However, for HII region there is a possiblity for bremsstralung radiation to be can be partially polarized by Thomson scattering however, this is a small effect and is not expected to polarise more than $10 \%$ ([Keating et al.] $1997$).

    \item The polarization of dust is not well known. In principle, emission from dust particles could be highly polarized, however [Hildebrand \& Dragovan] ($1995$) find that in their observations the majority of dust is polarized at the  level at  m with a small fraction of regions approaching  polarization. Moreover [Keating et al.] ($1997$) show that even at $100 \%$ polarization, extrapolation of the IRAS $100$
     m map with the COBE FIRAS index shows that dust emission is negligible below 80GHz. At higher frequencies it will become the dominant foreground.
\end{enumerate}


Synchrotron radiation is 
\section{Rough Notes}
Krachmalnicoff et al. (2018) have characterized the synchrotron polarized foreground emission analysing maps of the
southern sky from S-PASS at 2.3 GHz. Comparison with our
synchrotron results in Fig. 9 is not immediate because power
spectra are not measured over the same sky regions. Further, the
signal to noise ratio of the S-PASS data for synchrotron emission is larger than that of WMAP and Planck, which is a critical
advantage in characterizing the faint polarization signal at high
Galactic latitude. However, contamination by Faraday rotation
is likely to be significant for their largest sky regions extending
down to Galactic latitude |b| = 20◦
\end{document}
